\documentclass[12pt,a4paper]{report}
\usepackage[latin1]{inputenc}
\usepackage{amsmath}
\usepackage{amsfonts}
\usepackage{amssymb}
\usepackage[english]{babel}
\usepackage{ae}	%sostituisce i font raster con font vettoriali
\usepackage{pdfpages}
\usepackage{graphicx}
\usepackage{titling}
\newcommand{\subtitle}[1]{%
  \posttitle{%
    \par\end{center}
    \begin{center}\large#1\end{center}
    \vskip0.5em}%
}

% declare the path(s) where your graphic files are
%\graphicspath{./figures/}
% and their extensions so you won't have to specify these with
% every instance of \includegraphics
\DeclareGraphicsExtensions{.pdf,.jpeg,.png}


\author{Augusto Nascetti}
\title{\textbf{ABACUS}\\\textbf{A}dvanced \textbf{B}oard for \textbf{A}ctive \textbf{C}ontrol\\of \textbf{U}niversitary \textbf{S}atellites}
\subtitle{User Manual}
\date{March 2014}    
\begin{document}

\maketitle

\begin{titlepage}
 \begin{center}
%     \includegraphics[width=4cm]{logosapienza.pdf}\\
     \vspace{1em}
     {\Large \textsc{ABACUS}}\\
     \vspace{1em}
     {\Large \textsc{User Manual}}\\
     \vspace{2em}
     {\normalsize Rev 1.0}\\
     \vspace{1em}
     {\normalsize Augusto Nascetti}\\
     \vspace{1em}
 \end{center}
\end{titlepage}

\tableofcontents

\chapter{Introduction}
ABACUS is an on-board computer suitable for nano- and micro-satellites based on a fault-tolerant design and a reliability-enhancing architecture. ABACUS implements hardware redundancy, provided by a microcontroller and an FPGA, distributed in two independent but cooperative sections: technology diversity grants common mode fault tolerance. The microcontroller ensures low power operation and gives a simple access to buses, memories and peripherals. The FPGA offers all the advantages of the RTL coding for implementing fast error detection and correction techniques, data coding, state-machines or extra communication interfaces, supporting specific satellite operations. Alternatively the FPGA can be used to develop an entire application using e.g. third part IP cores, like microprocessors and microcontrollers. The microcontroller and the FPGA sections are powered from the common 5V power bus but with independent power regulation that can be switched off independently in order to limit power consumption and introduce overheat protection. The two sections share the same I2C system bus and several general purpose I/O (GPIO) lines for data exchange and synchronization. Each section has its own memory system for data storage. 
Several on-board sensors give housekeeping and health monitoring. Two sensors measure both FPGA and microcontroller temperature; a current shunt amplifier returns the overall current consumption. One magnetometer, one gyroscope and one accelerometer, provide information about satellite attitude.
The board has a PC104 form factor and its pinout is compatible with the cubesat standard.

\chapter{Overview}
Bla bla bla

\chapter{Specifications}
Specifications

\chapter{System Description}
General description
\section{Microcontroller section}
Microcontroller
\section{FPGA section}
FPGA
\section{Sensors section}
Sensors

\section{Board configuration}
Configuration

\subsection{Ext GPIO}
GPIO

\subsection{Differential UART port}
422 485

\subsection{RTCIRQ}
RTCIRQ

\chapter{Interfacing}
interface overview

\section{PC-104 connector pinout}
Connector

\begin{figure}
\centering
\includegraphics[width=\textwidth]{./figures/H1H2.pdf}
%\vspace{2.5in}
\caption{H1, H2 pinout overview}
\label{fig_H1H2}
\end{figure}

\begin{figure}
\centering
\includegraphics[height=0.9\textheight]{./figures/H1H2view.pdf}
%\vspace{2.5in}
\caption{H1, H2 pins}
\label{fig_H1H2view}
\end{figure}


\section{UART0 connector pinout}
Connector

\section{UART2 connector pinout}
Connector

\section{FPGA GPIO and IP connector}
Connector

\section{Microcontroller JTAG programming connector}
Connector
Hardware needed

\section{FPGA JTAG programming connector}
Connector
Hardware needed

\chapter{OBC programming}
Overview

\section{Custom FW development}
\subsection{Microcontroller}
Details (connections, chip pinout, ref to tools etc)

\subsection{FPGA}
Details, ref to ISE etc

\section{Using ABACUS dedicated libraries}
Intro
Lib overview (fig)
\subsection{BoardConfig}
\subsection{UART}
\subsection{I2C}
\subsection{SPI}
\subsection{Flash Memory}
\subsection{Sensors}
\subsection{Real Time Clock}
\subsection{FPGA data exchange}
\subsection{External GPIO}

\end{document}